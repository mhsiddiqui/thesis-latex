\chapter{Methodology}

Text to Speech synthesis system designed in this paper is divided in two sub modules. One perform analysis and preprocessing of 
data and other transform processed data into sound signals. These modules are following

\begin{enumerate}
  \item Text Processing Unit
  \item Speech Synthesis System
\end{enumerate}

\section{Text Processing Unit}
This unit is the unit which is responsible for processing of text before text is sent to speech synthesis system. 
This unit find numbers, dates and time in input data and converts it into format acceptable by speech synthesizer.  
This module consists of following sub modules.

\begin{enumerate}
  \item Special Character Processor
  \item Semantic Tagger  
  \item Text Generator
  \item Text Formatter 
\end{enumerate}


\subsection{Special Character Processor}

Raw text may contain special characters such as punctuation marks. These characters helps to understand context of a word but 
are not converted into sounds. We remove all such characters from text before further processing. 
Another processing which is done is conversion of all arabic numerals like \textarabic{١}, \textarabic{٢} and \textarabic{٣} into their corresponding 
characters like 1, 2 and 3. It is because it makes it easy to further process text after it has been converted into same type of numerals.

\subsection{Semantic Tagger}
The purpose of Semantic Tagger is to identify numbers, dates and time from input data and give them proper tags. 
All numbers in input data are converted into arabic numerals of type 1, 2 and 3 in previous step. 
There can be multiple form of numbers, dates and time. These forms are explained below.


\begin{enumerate}    
  \item Dates in following format

  \begin{enumerate}[label=\alph*.]
    \item 12/11/2018 or 12/11/18 with different separaters like \enquote{/} or \enquote{-} or \enquote{.}
    \item \texturdu{12 دسمبر 2012}
    \item \texturdu{12 دسمبر}
  \end{enumerate}
  \item Number in following format
  \begin{enumerate}[label=\alph*.]
    \item Whole Numbers such as 123
    \item Floating point numbers such as 12.3
  \end{enumerate}

  \item Time in following format

  \begin{enumerate}[label=\alph*.]
    \item 12:12
    \item 12:12:12
  \end{enumerate}
\end{enumerate}


In table \ref{table:semantic_tagger_regex}, regex used for identification of these numbers, dates and time are shown.

\begin{table}[]
\centering
\resizebox{\textwidth}{!}{%
\begin{tabular}{|c|c|c|}
\hline
\textbf{Regex} & \textbf{Type} & \textbf{Example} \\ \hline
(\textbackslash{}d+(?:\textbackslash{}.\textbackslash{}d+)?) & Integer or Floating point number & 123 or 12.312 \\ \hline
\textbackslash{}d\{1,2\}:\textbackslash{}d\{1,2\}(?::\textbackslash{}d\{1,2\})? & Time with or without seconds & 5:12 or 5:12:10 \\ \hline
\textbackslash{}d\{1,4\}{[}./-{]}\textbackslash{}d\{1,4\}{[}./-{]}\textbackslash{}d\{1,4\} & Date with separator like \enquote{/} or \enquote{-} or \enquote{.} & 12-10-2018 or 12/10/18 \\ \hline
\%s \textbackslash{}d\{4\} & Dates with month name in Urdu. This is checked by replacing \%s with each month name separately. & \texturdu{12 دسمبر} \\ \hline
\end{tabular}%
}
\caption{Regular Expression for Semantic Tagger}
\label{table:semantic_tagger_regex}
\end{table}