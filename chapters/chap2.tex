\chapter{Related Work}

The TTS conversion is not a new field and people have been working on this field before electronic
signal processing techniques. In beginning, people tried to build machines which were used to
create human sound. After the development of computers, better systems were built using different
techniques. In \cite{swetha2013text}, basic speech synthesizing technique
which works by concatenation of small recorded speech segments called phonemes to form
complete speech are discussed. Each word is first divided into syllables and then pronunciation for each syllable
is concatenated in order to get pronunciation for whole word. This concatenated word has some
delay between pronunciations of each syllable which is removed and as a result, final
pronunciation of that word is obtained. Problems like Text Preprocessing, Pronunciation and
Prosody makes it difficult. In Text Preprocessing, digits and abbreviations are converted to full
words. Other problem is guessing correct pronunciation of a word. For example, word “lives” has
different pronunciation in “He lives in Lahore” and “He saved two lives”. To create naturalness in
sound, stress and intonation are applied to the input text which is also a very complex task.


A rule based technique is designed in \cite{elovitz1976automatic}. The dataset is developed by extracting
50,000 words from standard Corpus, Corpus of Present-Day Edited American English i.e. Brown
corpus \cite{ku1967computational}. The system gave accuracy of about 93%. A more
improved system was proposed in \cite{carlson1982multi} and \cite{klatt1982klattalk}.


Dictionary and rule based approach is used in \cite{liberman1992text} where 1000,000 words were used for training model. A TTS system with
prosody and concatenative speech parameters that were extracted through use of probabilistic
learning methods in \cite{huang1996whistler}. A formant and Concatenative synthesis is developed in \cite{huang1997recent}
where small segments of phonemes were concatenated to form whole speech. A training database
was used contains about 6,000 phonetically balanced sentences recorded in natural style. In \cite{hunt1996unit}, linear regression and unit selection 
based speech synthesis is designed using ATR Japanese database.


Statistical parametric speech synthesis is another approach which uses parameters to describe
speech. In this technique, model is learned from speech data. This technique works better than
concatenative technique. In \cite{merritt2013investigating}, author discussed shortcomings of concatenative
synthesis and why Hidden Markov model based technique is better than concatenative.

A Hidden Markov model based Statistical parametric speech synthesis system was developed in
\cite{yoshimura1998duration} where 450 sentences from ATR Japanese Database were used.  A similar system is designed in 
\cite{tokuda2000speech} using Hidden Markov Model and evaluated it by taking input
from Japanese database and by generating feature vector of those sentences and compared with
original speech.


A Hidden Markov model and unit selection based system is proposed in \cite{tokuda2002hmm} in which
524 sentences from CMU communicator database were taken. The sentences are used to develop
system and two algorithms are compared. The result showed unit give high quality results but it
uses large data, voice remain fixed whereas HTS gave smooth, stable, various voices but it gives
buzzy speech. In \cite{harashima2006review}, Hidden Markov model and rule based approach is applied 
on voices taken from e-learning courses and online lessons for dataset creation and tested by
generating voices and given as input to students to interpret it.


Corpus based approach for Expressive Prosody Modeling is applied in \cite{eide2004corpus} in which
manually produced dataset which is not publicly available was used. To evaluate the synthesized
speech, the output is given for testing to 32 native English speakers. For bad news, good news and
for yes/no the accuracy is 70.2%, 80.3% and 84% respectively


Tones and Break Indices ToBI are discussed on American English in \cite{pitrelli2004tobi}. Majorly
bi-gram and tri-gram were used to predict the occurrences of particular word or letter. Analysis
was performed by multiple techniques. The corpus was divided into following five yes-no
questions, either-or questions, other questions (hereafter, “wh-questions”), exclamations, and other
declarative sentences categories for the analysis purpose. Analysis by word frequency count
produced not good results and the reason was that the data was of variant types. The results show
that professional speakers produces better and informative prosodic events as compared to
ordinary speakers.

A TTS system for Azerbhaijani language using concatenative
synthesis in \cite{aida2010main} where small recordings were concatenated to make speech waveform.


Hidden Markov model based approach is used in \cite{baloyi2012text} to construct
speech synthesizer for Xitsonga which is an African language. The system received acceptability
of 92.3%.


TTS system for Fon language is designed in \cite{dagba2014text} using Multisyn algorithm
which consists of Natural Language Processing (NLP) and Digital Signal Processing (DSP)
modules. NLP consists of segmentation, Letter-to-Sound conversion and back-off rules module.
Back-off rules are applied when input text contains some characters that are not in us know
characters. DSP module than choose required unit from database of units are concatenate them to
form complete speech signals.


In \cite{ganai2016text}, hybrid text to speech converter is developed by
concatenating benefits of HMM based TTS system and waveform based TTS system. System is
developed using Matlab, a library of phoneme and their sound is created and waveform of audio
and audio itself is generated using these libraries.

% Speech to text system is becoming more and more important for research because speech is getting
% more common for communication between computer/machines and also systems that use speech
% are faster and less exposed to error. As TTS system increases efficiency of machines extensive
% research on these systems is going own from last few decades. A text to speech system is a device
% that converts text to spoken output without any human interaction. Different systems are developed
% to transform text to ASCI code and then to speech for English, systems for other languages are also
% designed in \cite{carlson1982multi}.





















