\chapter{Experiments and Results}

The purpose of a Text to Speech system is to build a system which is capable of generating voice as close to human voice as possible. 
The generated voice should be intelligible so that people can easily understand generated voice. To find quality of generated sound, every speech synthesis 
system is evaluated. The evaluation process can be subjective as weill as objective. In subjective evaluation, system is evaluated using human users while in objective 
evaluation, different algorithms are used. For our system, we only focussed on subjective testing. There are many type of tests. Some of them are listed below.

\begin{itemize}
  \item{\textbf{Diagnostic Rhyme Test (DRT)}:} Tests for words which are similar in sound but differ with each other in initial consonants \cite{voiers1977diagnostic}.
  \item{\textbf{Modified Diagnostic Rhyme Test (M-DRT)}:} Tests for words which are similar in sound but differ with each other in last consonants \cite{house1965articulation}.
  \item{\textbf{Naturalness Test}:} Test to evaluate whether generated voice is close to human voice.
  \item{\textbf{intelligiblity Test}:} Test to evaluate whether generated voice is understandable.
  \item{\textbf{Usability Test}:} Test to evaluate whether generated voice can be used for blind or non-blind people.
\end{itemize}