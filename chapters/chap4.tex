\chapter{Experiments and Results}

The purpose of a Text to Speech system is to build a system which is capable of generating voice as close to human voice as possible. 
The generated voice should be intelligible so that people can easily understand generated voice. To find quality of generated sound, every speech synthesis 
system is evaluated. The evaluation process can be subjective as weill as objective. In subjective evaluation, system is evaluated using human users while in objective 
evaluation, different algorithms are used. For the process of evaluation, native speaker of specific language are required. For our system, we only focussed on subjective testing. 

\section{Subjective Testing}

There are many type of subjective tests. Some of them are listed below.

\begin{itemize}
  \item Diagnostic Rhyme Test (DRT)
  \item Modified Diagnostic Rhyme Test (M-DRT)
  \item Naturalness Test
  \item Intelligibility Test
  \item Usability Test
\end{itemize}

\subsection{Diagnostic Rhyme Test (DRT)}
This test is for Indicative and relative assessment of the understandability of single starting consonants. Test is conducted with words which are similar in sound but differ with each other in initial consonants \cite{voiers1977diagnostic}. User have to listen speech generated by system of a specific word and identify that spoken word from list of words. The result of this test is the percentage of words correctly identified.

\subsection{Modified Diagnostic Rhyme Test (M-DRT)}
This is to test demonstrative and relative assessment of the coherence of single last consonants.Tests is conducted using words which are similar in sound but differ with each other in last consonants \cite{house1965articulation}. User have to listen speech generated by system of a specific word and identify that spoken word from list of words. The result of this test is the percentage of words correctly identified.

\subsection{Naturalness Test}
This test is conducted to find out to which extent generated voice is close to human voice. The is conducted by rating generated voice from 1 to 5. User will play some voice and will give synthesized speech some value from 1 to 5 according to his understanding of the speech.

\subsection{Intelligibility Test}
This test is conducted to find out to which extent generated voice is understandable. The is conducted by rating generated voice from 1 to 5. User will play some voice and will give synthesized speech some value from 1 to 5 according to his understanding of the speech.

\subsection{Usability Test}
This test is conducted to find out to which extent generated voice can be used for blind or non-blind people. The is conducted by rating generated voice from 1 to 5. User will play some voice and will give synthesized speech some value from 1 to 5 according to his understanding of the speech.


\section{Evaluation}
For the process of evaluation, we selected list of 64 words and 8 sentences. Words are selected on the sound and first and last words in order to use in Diagnostic Rhyme Test (DRT) and Modified Diagnostic Rhyme Test (M-DRT).

\subsection{Methodology}
An evaluation form is designed which have three sections. 

\begin{itemize}
  \item Diagnostic Rhyme Test (DRT)
  \item Modified Diagnostic Rhyme Test (M-DRT)
  \item Mean Opinion Score (MOS)
\end{itemize}

