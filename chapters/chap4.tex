\chapter{Experiments and Results}

The reason for a Text to Speech system is to manufacture a system which is equipped for producing voice as near human voice as could reasonably be expected. The generated voice should be intelligible so that people can easily understand generated voice. To find quality of generated sound, every speech synthesis system is evaluated. The evaluation process can be subjective as well as objective. In subjective evaluation, system is evaluated using human users while in objective evaluation, different algorithms are used. For the process of evaluation, native speaker of specific language is required. For our system, we only focused on subjective testing.

\section{Subjective Testing}

There are many type of subjective tests. Some of them are listed below.

\begin{itemize}
  \item Diagnostic Rhyme Test (DRT)
  \item Modified Diagnostic Rhyme Test (M-DRT)
  \item Naturalness Test
  \item Intelligibility Test
  \item Usability Test
\end{itemize}

\subsection{Diagnostic Rhyme Test (DRT)}
DRT is performed in order to do indicative and relative assessment of the understandability of single starting consonants. This is conducted with words which are similar in sound but differ with each other in initial consonants \cite{voiers1977diagnostic}. User has to listen speech generated by system of a specific word and identify that spoken word from list of words. The result of this test is the percentage of words correctly identified.

\subsection{Modified Diagnostic Rhyme Test (M-DRT)}
This test is to check demonstrative and relative assessment of the coherence of single last consonants. This is conducted using words which are similar in sound but differ with each other in last consonants \cite{house1965articulation}. User has to listen speech generated by system of a specific word and identify that spoken word from list of words. The result of this test is the percentage of words correctly identified.

\subsection{Naturalness Test}
This test is directed to discover to which degree created voice is near human voice. This is conducted by rating generated voice from 1 to 5. User will play some voice and will give synthesized speech some value from 1 to 5 according to his understanding of the speech.

\subsection{Intelligibility Test}
This test is conducted to find out to which extent generated voice is understandable. This is conducted by rating generated voice from 1 to 5. User will play some voice and will give synthesized speech some value from 1 to 5 according to his understanding of the speech.

\subsection{Usability Test}
This test is conducted to find out to which extent generated voice can be used for blind or non-blind people. This is conducted by rating generated voice from 1 to 5. User will play some voice and will give synthesized speech some value from 1 to 5 according to his understanding of the speech.


\section{Evaluation}
For the process of evaluation, we selected list of 64 words and 8 sentences. Words are selected on the sound and first and last words in order to use in DRT and M-DRT.

\subsection{Methodology}
An evaluation form is designed which have three sections. 


\subsubsection{DRT Section}

This section has 8 questions. In each question, user will play recording of some words which are converted to sound using our TTS system. These words are tested through following carrier sentence.

  \texturdu{نیچے دیے گئے الفاظ میں سے حرکت پے نشان لگائیں}

User will have to select that word from list of words which have same sound but different first character.

\subsubsection{M-DRT Section}

This section has 8 questions. In each question, user will play recording of some words which are converted to sound using our TTS system. These words are tested through following carrier sentence.

  \texturdu{نیچے دیے گئے الفاظ میں سے حرکت پے نشان لگائیں}

User will have to select that word from list of words which have same sound but different last character.

\subsubsection{Mean Opinion Score (MOS) Section}
In this section, user will have to play a sentence and user will rate converted text from 1 to 5. User will have to rate them on the basis of following properties.

\begin{itemize}
  \item {\textbf{Naturalness}:} How much converted sound is near sound delivered by a human?
  \item {\textbf{Intelligibility}:} How conveniently the word was perceived?
  \item {\textbf{Overall}:} How do you rate this sound overall? Is this system is usable to use for blind people?
\end{itemize}

System is evaluated by 47 (33 males and 14 female) native Urdu speakers who carefully listened and evaluated system. Each listener evaluated each question separately after listening it.

\section{Results}

The results of evaluation were very satisfying as it is observed that most of the words which have same sound are easily recognizable. Almost 90\% of the such words are correctly identified by users. The result shows that output of the system is recognizable and intelligible but not very natural. Table \ref{table:evaluation_result} shows the complete result of evaluation.

\begin{table}[]
\centering
\resizebox{\textwidth}{!}{%
\begin{tabular}{|c|c|}
\hline
\textbf{Test}                                  & \textbf{Score} \\ \hline
Diagnostic Rhyme Test (DRT)                    & 0.95           \\ \hline
Modified Diagnostic Rhyme Test (M-DRT) Section & 0.88           \\ \hline
Naturalness                                    & 3.24           \\ \hline
Intelligibility                                & 3.42           \\ \hline
Usability                                      & 3.49           \\ \hline
\end{tabular}%
}
\caption{Evaluation Result}
\label{table:evaluation_result}
\end{table}