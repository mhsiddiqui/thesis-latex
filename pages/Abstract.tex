
{\LARGE\textbf {Abstract}} \\ % \\ = new line
The process of transformation of data from textual form into voice output is called speech synthesis. The type of system which plays out this assignment is known as a text to speech (TTS) system or sometimes as speech synthesizer. The synthesized speech sometimes referred as artificial speech. This system can assist individuals with various handicaps like visual weakness in their day by day undertakings and furthermore help in inter language correspondence. These systems are being used in screen reading software, games, animations, machine to machine communication and may other applications. Artificial speech can be synthesized by concatenation of small segments of speech series of words. Formant synthesis and unit selection synthesis are based on this types of synthesis system. Another popular technique which is being used in most recent couple of years is statistical parametric speech synthesis. This technique when used with Hidden Markov (HMM) or Deep Neural Networks (DNN) gives extremely promising results.

In this paper, development of statistical parametric speech synthesis system using Festival TTS for Urdu is discussed. HMM based TTS system is developed using 70 minutes of speech data. This paper divides complete process of speech synthesis in two sub processes i.e. text preprocessing and speech synthesis. Text preprocessor system identifies and processes special characters, numbers (Arabic numerals, floating point and whole numbers), time and dates text in input data. Speech synthesis process takes processed input and converts it into corresponding voice. In this process, Lexicon and letter to sound rules of Hindi are used. In the last step, performance of system evaluated using Diagnostic Rhyme Test (DRT), Modified Diagnostic Rhyme Test (M-DRT), Naturalness Test, Intelligibility Test and Usability Test.


\\ \\ \\
\textbf{Keywords:}
Text to Speech, Urdu Text Preprocessor, Hidden Markov model, Festival, Festvox
