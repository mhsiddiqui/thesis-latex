
{\LARGE\textbf {Abstract}} \\ % \\ = new line

The process of transformation of textual data into voice output is called speech synthesis. The system which perform this task is called text to speech (TTS) system or speech synthesizer. In this paper, development of statistical parametric speech synthesis system for Urdu using Festival TTS is discussed. Speech corpus used for the development of the system consists of 70 minutes recording produced by CLE and has free license for public use. This system is using Letter to Sound rules and Word Pronunciation rules of Hindi TTS due to unavailability of these components for Urdu language. Phone set of Urdu are mapped to their corresponding Hindi phone set which is then processed according to Hindi language lexicon. Speech parameters i.e. F0, MFCCs and duration are extracted using labeled speech data which is used to train and build the model using Hidden Markov Model (HMM). To improve the quality of speech generated by the system, an NER and token processing system is built which identifies date, time and numeric values on the basis of context and convert them into corresponding Urdu text. System is evaluated using Diagnostic Rhyme Test (DRT), Modified Diagnostic Rhyme Test (M-DRT), Naturalness Test, Intelligibility Test and Usability Test on the basis of feedback taken from 47 native Urdu language speakers.

\\ \\ \\
\textbf{Keywords:}
Text to Speech, Urdu Text Preprocessor, Hidden Markov model, Festival, Festvox, Urdu Text to Speech
