\chapter{Tables}

\begin{longtable}[c]{|c|c|}
\hline
\textbf{Number} & \textbf{Mapping}    \\ \hline
\endhead
%
.               & \texturdu{اعشاریہ}  \\ \hline
0               & \texturdu{زیرو}     \\ \hline
1               & \texturdu{ایک}      \\ \hline
2               & \texturdu{دو}       \\ \hline
3               & \texturdu{تین}      \\ \hline
4               & \texturdu{چار}      \\ \hline
5               & \texturdu{پانچ}     \\ \hline
6               & \texturdu{چھ}       \\ \hline
7               & \texturdu{سات}      \\ \hline
8               & \texturdu{آٹھ}      \\ \hline
9               & \texturdu{نو}       \\ \hline
10              & \texturdu{دس}       \\ \hline
11              & \texturdu{گیارہ}    \\ \hline
12              & \texturdu{بارہ}     \\ \hline
13              & \texturdu{تیرہ}     \\ \hline
14              & \texturdu{چودہ}     \\ \hline
15              & \texturdu{پندرہ}    \\ \hline
16              & \texturdu{سولہ}     \\ \hline
17              & \texturdu{سترہ}     \\ \hline
18              & \texturdu{اٹھارہ}   \\ \hline
19              & \texturdu{انیس}     \\ \hline
20              & \texturdu{بیس}      \\ \hline
21              & \texturdu{اکیس}     \\ \hline
22              & \texturdu{بائیس}    \\ \hline
23              & \texturdu{تئیس}     \\ \hline
24              & \texturdu{چوبیس}    \\ \hline
25              & \texturdu{پچیس}     \\ \hline
26              & \texturdu{چھبیس}    \\ \hline
27              & \texturdu{ستائیس}   \\ \hline
28              & \texturdu{اٹھائیس}  \\ \hline
29              & \texturdu{انتیس}    \\ \hline
30              & \texturdu{تیس}      \\ \hline
31              & \texturdu{اکتیس}    \\ \hline
32              & \texturdu{بتیس}     \\ \hline
33              & \texturdu{تینتیس}   \\ \hline
34              & \texturdu{چونتیس}   \\ \hline
35              & \texturdu{پینتیس}   \\ \hline
36              & \texturdu{چھتیس}    \\ \hline
37              & \texturdu{سینتیس}   \\ \hline
38              & \texturdu{اٹھتیس}   \\ \hline
39              & \texturdu{انتالیس}  \\ \hline
40              & \texturdu{چالیس}    \\ \hline
41              & \texturdu{اکتالیس}  \\ \hline
42              & \texturdu{بیالیس}   \\ \hline
43              & \texturdu{تینتالیس} \\ \hline
44              & \texturdu{چوالیس}   \\ \hline
45              & \texturdu{پینتالیس} \\ \hline
46              & \texturdu{چھیالیس}  \\ \hline
47              & \texturdu{سینتالیس} \\ \hline
48              & \texturdu{اڑتالیس}  \\ \hline
49              & \texturdu{انچاس}    \\ \hline
50              & \texturdu{پچاس}     \\ \hline
51              & \texturdu{اکیاون}   \\ \hline
52              & \texturdu{باون}     \\ \hline
53              & \texturdu{تریپن}    \\ \hline
54              & \texturdu{چون}      \\ \hline
55              & \texturdu{پچپن}     \\ \hline
56              & \texturdu{چھپن}     \\ \hline
57              & \texturdu{ستاون}    \\ \hline
58              & \texturdu{اٹھاون}   \\ \hline
59              & \texturdu{انسٹھ}    \\ \hline
60              & \texturdu{ساٹھ}     \\ \hline
61              & \texturdu{اکسٹھ}    \\ \hline
62              & \texturdu{باسٹھ}    \\ \hline
63              & \texturdu{تریسٹھ}   \\ \hline
64              & \texturdu{چونسٹھ }  \\ \hline
65              & \texturdu{پینسٹھ}   \\ \hline
66              & \texturdu{چھیاسٹھ}  \\ \hline
67              & \texturdu{ستاسٹھ}   \\ \hline
68              & \texturdu{اٹھاسٹھ}  \\ \hline
69              & \texturdu{انهتر}    \\ \hline
70              & \texturdu{ستر}      \\ \hline
71              & \texturdu{اکھتر}    \\ \hline
72              & \texturdu{بھتر}     \\ \hline
73              & \texturdu{تھتر}     \\ \hline
74              & \texturdu{چوہتر}    \\ \hline
75              & \texturdu{پچھتر}    \\ \hline
76              & \texturdu{چھہتر}    \\ \hline
77              & \texturdu{ستتر}     \\ \hline
78              & \texturdu{اٹھتر}    \\ \hline
79              & \texturdu{اناسی}    \\ \hline
80              & \texturdu{اسی}      \\ \hline
81              & \texturdu{اکاسی}    \\ \hline
82              & \texturdu{بیاسی}    \\ \hline
83              & \texturdu{تراسی}    \\ \hline
84              & \texturdu{چوراسی}   \\ \hline
85              & \texturdu{پچاسی}   \\ \hline
86              & \texturdu{چھیاسی}   \\ \hline
87              & \texturdu{ستاسی}    \\ \hline
88              & \texturdu{اٹھاسی}   \\ \hline
89              & \texturdu{نواسی}    \\ \hline
90              & \texturdu{نوے}      \\ \hline
91              & \texturdu{اکانوے}   \\ \hline
92              & \texturdu{بانوے}    \\ \hline
93              & \texturdu{ترانوے}   \\ \hline
94              & \texturdu{چورانوے}  \\ \hline
95              & \texturdu{پچانوے}   \\ \hline
96              & \texturdu{چھیانوے}  \\ \hline
97              & \texturdu{ستانوے}   \\ \hline
98              & \texturdu{اٹھانوے}  \\ \hline
99              & \texturdu{ننانوے}   \\ \hline
100             & \texturdu{سو}       \\ \hline
1000            & \texturdu{ھزار}     \\ \hline
100000          & \texturdu{لاکھ  }   \\ \hline
10000000        & \texturdu{کروڑ }    \\ \hline
1000000000      & \texturdu{ارب }     \\ \hline
100000000000    & \texturdu{کھرب }    \\ \hline
\caption{Number Mappings}
\label{table:number_mapping}\\
\end{longtable} 

\begin{longtable}[c]{|c|c|}
\hline
\textbf{Number} & \textbf{Mapping} \\ \hline
\endhead
%
january         & \texturdu{جنوری}            \\ \hline
february        & \texturdu{فروری }           \\ \hline
march           & \texturdu{مارچ }            \\ \hline
april           & \texturdu{اپریل}            \\ \hline
may             & \texturdu{مئی}              \\ \hline
june            & \texturdu{جون }             \\ \hline
july            & \texturdu{جولائی}           \\ \hline
august          & \texturdu{اگست}             \\ \hline
september       & \texturdu{ستمبر }           \\ \hline
october         & \texturdu{اکتوبر }          \\ \hline
november        & \texturdu{نومبر }           \\ \hline
december        & \texturdu{دسمبر}            \\ \hline
1               & \texturdu{جنوری}            \\ \hline
2               & \texturdu{فروری }           \\ \hline
3               & \texturdu{مارچ }            \\ \hline
4               & \texturdu{اپریل}            \\ \hline
5               & \texturdu{مئی}              \\ \hline
6               & \texturdu{جون }             \\ \hline
7               & \texturdu{جولائی}           \\ \hline
8               & \texturdu{اگست}             \\ \hline
9               & \texturdu{ستمبر }           \\ \hline
10              & \texturdu{اکتوبر }          \\ \hline
11              & \texturdu{نومبر }           \\ \hline
12              & \texturdu{دسمبر}            \\ \hline
\caption{Month Mapping}
\label{table:month_mapping}
\end{longtable}

% Hindi to Urdu Mapping

% Please add the following required packages to your document preamble:
% \usepackage{longtable}
% Note: It may be necessary to compile the document several times to get a multi-page table to line up properly
\begin{longtable}[c]{|c|c|c|}
\hline
\textbf{Hindi Character} & \textbf{Urdu Mapping} & \textbf{Character Detail} \\ \hline
\endhead
%
\textsanskrit{ँ}               & \texturdu{ں}            & Noon Ghunna                 \\ \hline
\textsanskrit{ऄ}               & \texturdu{َ}            & Arabic Zabar or Fatha       \\ \hline
\textsanskrit{न}               & \texturdu{ً}            & Arabic Fathatan             \\ \hline
\textsanskrit{अ}               & \texturdu{ا}            & Alif                        \\ \hline
\textsanskrit{ऒ}               & \texturdu{ا}            & Alif                        \\ \hline
\textsanskrit{अ}               & \texturdu{ء}            & Hamza                       \\ \hline
\textsanskrit{अ}               & \texturdu{ٔ}            & Hamza Above                 \\ \hline
\textsanskrit{अ}               & \texturdu{ع}            & Ain                         \\ \hline
\textsanskrit{आ}               & \texturdu{آ}            & Alif Madda                  \\ \hline
\textsanskrit{इ}               & \texturdu{ِ}            & Arabic Kasra or Zair        \\ \hline
\textsanskrit{ई}               & \texturdu{ی}            & Yeh                         \\ \hline
\textsanskrit{उ}               & \texturdu{ُ }           & Arabic Damma or Paish       \\ \hline
\textsanskrit{ू}               & \texturdu{ؤ}            & Waw with hamza above        \\ \hline
\textsanskrit{ऊ}               & \texturdu{ؤ}            & Waw with hamza above        \\ \hline
\textsanskrit{ऋ}               & \texturdu{رِ}           & Reh with Zair               \\ \hline
\textsanskrit{ए}               & \texturdu{ے}            & Baree Yeh                   \\ \hline
\textsanskrit{ऐ}               & \texturdu{آے}           & Aaey                        \\ \hline
\textsanskrit{ओ}               & \texturdu{ؤ}            & Waw with hamza above        \\ \hline
\textsanskrit{औ}               & \texturdu{آو}           & Aao                         \\ \hline
\textsanskrit{क़}               & \texturdu{ق}            & Qaf                         \\ \hline
\textsanskrit{क}               & \texturdu{ک}            & Kaaf                        \\ \hline
\textsanskrit{ख}               & \texturdu{کھ‍}          & Khay                        \\ \hline
\textsanskrit{ख़}               & \texturdu{خ}            & Khay                        \\ \hline
\textsanskrit{ग}               & \texturdu{گ}            & Gaaf                        \\ \hline
\textsanskrit{घ}               & \texturdu{گھ‍}          & Ghaa                        \\ \hline
\textsanskrit{घ}               & \texturdu{چ}            & Chay                        \\ \hline
\textsanskrit{छ}               & \texturdu{چھ}           & Chhay                       \\ \hline
\textsanskrit{ज}               & \texturdu{ج}            & Jeem                        \\ \hline
\textsanskrit{झ}               & \texturdu{جھ}           & Jhay                        \\ \hline
\textsanskrit{ञ}               & \texturdu{ياں}          & Yaan                        \\ \hline
\textsanskrit{ट}               & \texturdu{ٹ}            & Tay                         \\ \hline
\textsanskrit{ठ}               & \texturdu{ٹھ‍}          & Thay                        \\ \hline
\textsanskrit{ड}               & \texturdu{ڈ}            & Daal                        \\ \hline
\textsanskrit{ढ}               & \texturdu{ڈھ‍}          & Dhaal                       \\ \hline
\textsanskrit{ण}               & \texturdu{ڈاں}          & Daan                        \\ \hline
\textsanskrit{त}               & \texturdu{ت}            & Tay                         \\ \hline
\textsanskrit{त}               & \texturdu{ط}            & Toain                       \\ \hline
\textsanskrit{थ}               & \texturdu{تھ‍}          & Thay                        \\ \hline
\textsanskrit{द}               & \texturdu{د}            & Dal                         \\ \hline
\textsanskrit{ध}               & \texturdu{دھ‍}          & Dhal                        \\ \hline
\textsanskrit{न}               & \texturdu{ن}            & Noon                        \\ \hline
\textsanskrit{प}               & \texturdu{پ}            & Pay                         \\ \hline
\textsanskrit{फ}               & \texturdu{پھ‍}          & Phay                        \\ \hline
\textsanskrit{ब}               & \texturdu{ب}            & Bay                         \\ \hline
\textsanskrit{ब}               & \texturdu{بھ}           & Bhay                        \\ \hline
\textsanskrit{म}               & \texturdu{م}            & Meem                        \\ \hline
\textsanskrit{र}               & \texturdu{ر}            & Ray                         \\ \hline
\textsanskrit{ल}               & \texturdu{ل}            & Laam                        \\ \hline
\textsanskrit{ऴ}               & \texturdu{ّ}            & Arabic Shadda               \\ \hline
\textsanskrit{व}               & \texturdu{و}            & Wow                         \\ \hline
\textsanskrit{श}               & \texturdu{ش}            & Sheen                       \\ \hline
\textsanskrit{स}               & \texturdu{ث}            & Say                         \\ \hline
\textsanskrit{स}               & \texturdu{س}            & Seen                        \\ \hline
\textsanskrit{स}               & \texturdu{ص}            & Saad                        \\ \hline
\textsanskrit{ह}               & \texturdu{ح}            & Hay                         \\ \hline
\textsanskrit{ह}               & \texturdu{ہ}            & Gol Heh                     \\ \hline
\textsanskrit{ह}               & \texturdu{ھ}            & Heh                         \\ \hline
\textsanskrit{ग़}               & \texturdu{غ}            & Ghain                       \\ \hline
\textsanskrit{ज़}               & \texturdu{ذ}            & Zaal                        \\ \hline
\textsanskrit{ज़}               & \texturdu{ز}            & Zay                         \\ \hline
\textsanskrit{ज़}               & \texturdu{ظ}            & Zoain                       \\ \hline
\textsanskrit{ज़}               & \texturdu{ژ}            & Zay                         \\ \hline
\textsanskrit{ज़}               & \texturdu{ض}            & Zaad                        \\ \hline
\textsanskrit{ड़}               & \texturdu{ڑ}            & Rhay                        \\ \hline
\textsanskrit{ढ़}               & \texturdu{ڑھ‍}          & Rhay                        \\ \hline
\textsanskrit{फ़}               & \texturdu{ف}            & Fay                         \\ \hline
\textsanskrit{य़}               & \texturdu{ئ}            & Hamza Choti Yeh             \\ \hline
\textsanskrit{०}               & 0            & Zero                 \\ \hline
\textsanskrit{१}               & 1            & One                  \\ \hline
\textsanskrit{२}               & 2            & Two                  \\ \hline
\textsanskrit{३}               & 3            & Three                \\ \hline
\textsanskrit{४}               & 4            & Four                 \\ \hline
\textsanskrit{५}               & 5            & Five                 \\ \hline
\textsanskrit{६}               & 6            & Six                  \\ \hline
\textsanskrit{७}               & 7            & Seven                \\ \hline
\textsanskrit{८}               & 8            & Eight                \\ \hline
\textsanskrit{९}               & 9            & Nine                 \\ \hline
\textsanskrit{०}               & \textarabic{٠}            & Arabic Zero                 \\ \hline
\textsanskrit{१}               & \textarabic{١}            & Arabic One                  \\ \hline
\textsanskrit{२}               & \textarabic{٢}            & Arabic Two                  \\ \hline
\textsanskrit{३}               & \textarabic{٣}            & Arabic Three                \\ \hline
\textsanskrit{४}               & \textarabic{٤}            & Arabic Four                 \\ \hline
\textsanskrit{५}               & \textarabic{٥}            & Arabic Five                 \\ \hline
\textsanskrit{६}               & \textarabic{٦}            & Arabic Six                  \\ \hline
\textsanskrit{७}               & \textarabic{٧}            & Arabic Seven                \\ \hline
\textsanskrit{८}               & \textarabic{٨}            & Arabic Eight                \\ \hline
\textsanskrit{९}               & \textarabic{٩}            & Arabic Nine                 \\ \hline
\caption{Hindi to Urdu Character Mappings}
\label{table:hindi_to_urdu}
\end{longtable}
